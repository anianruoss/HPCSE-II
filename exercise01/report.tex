\documentclass[11pt]{article}


\usepackage{amsmath}
\usepackage{booktabs}
\usepackage{cleveref}
\usepackage{enumerate}
\usepackage{float}
\usepackage{graphicx}
\usepackage{listings}

\setlength\parindent{0pt}

\begin{document}

    \title{HPCSE II - Exercise 1}
    \author{Anian Ruoss}
    \maketitle

    All experiments were run on a node obtained by running the following
    command:
    \begin{lstlisting}[language=bash, basicstyle=\footnotesize]
bsub -n 24 -R fullnode -R "select[model=XeonE5_2680v3]" -Is bash
    \end{lstlisting}

    \subsection*{Task 4}
    \label{subsec:Task4}
    \begin{enumerate}[a)]
        \item See the first row in Table~\ref{table:OptGridCount}.
        \item The optimal number of grids is $6$ as illustrated
        in Table~\ref{table:OptGridCount}.
        \begin{table}[H]
            \caption{Iterations and running times for different numbers of
            grids for DownRelaxations $=1$ and UpRelaxations $=1$.}
            \begin{center}
                \begin{tabular}{c|ccc}
\toprule
 GridCount & Iterations & Running Time & Verification \\
\midrule
         1 &     199204 &     150.165s &       Passed \\
         2 &      43522 &      57.425s &       Passed \\
         3 &      32457 &      46.268s &       Passed \\
         4 &      32102 &      46.524s &       Passed \\
         5 &      30625 &      44.926s &       Passed \\
         6 &      22973 &      33.810s &       Passed \\
         7 &      33171 &      48.768s &       Passed \\
         8 &      51762 &      75.945s &       Passed \\
\bottomrule
\end{tabular}

            \end{center}
            \label{table:OptGridCount}
        \end{table}
        \item The optimal combination of UpRelaxations and DownRelaxations is
        determined in a greedy way.
        The optimal number of DownRelaxations in terms of running time is $3$
        as illustrated in Table~\ref{table:OptDownRelax}.
        Although, values of $5$ and $7$ produce less iterations, the time per
        iteration is significantly higher.
        \begin{table}[H]
            \caption{Iterations and running times for different numbers of
            DownRelaxations given GridCount $= 6$ and UpRelaxations $= 1$.}
            \begin{center}
                \begin{tabular}{c|ccc}
\toprule
 DownRelaxations & Iterations & Running Time & Verification \\
\midrule
               1 &      22973 &      33.770s &       Passed \\
               2 &      37152 &      72.471s &       Passed \\
               3 &      13467 &      33.071s &       Passed \\
               4 &      20885 &      61.631s &       Passed \\
               5 &      12068 &      41.526s &       Passed \\
               6 &      14811 &      58.096s &       Passed \\
               7 &      10130 &      44.835s &       Passed \\
\bottomrule
\end{tabular}

            \end{center}
            \label{table:OptDownRelax}
        \end{table}
        The optimal number of UpRelaxations is $1$ as illustrated in
        Table~\ref{table:OptUpRelax}.
        \begin{table}[H]
            \caption{Iterations and running times for different numbers of
            UpRelaxations with GridCount $= 6$ and DownRelaxations $= 3$.}
            \begin{center}
                \begin{tabular}{c|ccc}
\toprule
 UpRelaxations & Iterations & Running Time & Verification \\
\midrule
             1 &      13467 &      33.055s &       Passed \\
             2 &      13467 &      34.066s &       Passed \\
             3 &      13467 &      35.200s &       Passed \\
             4 &      13467 &      36.909s &       Passed \\
             5 &      13467 &      37.349s &       Passed \\
             6 &      13467 &      38.467s &       Passed \\
             7 &      13467 &      39.613s &       Passed \\
\bottomrule
\end{tabular}

            \end{center}
            \label{table:OptUpRelax}
        \end{table}
        \item The optimal configuration with respect to the results from
        tables~\ref{table:OptGridCount},~\ref{table:OptDownRelax},
        and~\ref{table:OptUpRelax} is given by GridCount $= 6$,
        DownRelaxations $= 3$, and UpRelaxations $= 1$.
        To find the global minimum one would have to run a grid search on all
        three parameters.
        However, since the exercise asked for a greedy approach and since
        the local minimum performed fairly well, the grid search was omitted.
        \item The solution reaches its equilibrium state for low frequencies
        faster on coarser grids as every cell covers more area (it is also
        cheaper to compute a full pass on coarse grids).
        Larger frequencies can only be resolved on finer grids and thus
        the solution is alternatingly computed on finer and coarser grids to
        increase convergence speed while maintaining a small discretization
        error.
        \item Using more DownRelaxations increases the number of iterations
        in an alternating manner (the TA could not provide an explanation
        for this phenomenon during the exercise class).
        Furthermore, the running time decreases slightly from $1$ to $3$
        because at every grid level the solution can converge to its
        equilibrium state before passing on its values to the next grid.
        However, using more relaxation steps doesn't lead to a better
        convergence as it has already converged on the grid with fewer
        relaxations and thus just increases the time spent on every grid.
        The number of UpRelaxations doesn't seem to affect the iterations at
        all and just increases the time spent at every grid.
    \end{enumerate}

    \subsection*{Task 5}
    \label{subsec:Task5}

    \begin{enumerate}[a)]
        \item \textit{Loop Interchange} can be applied to a prevent bad
        memory access patterns, i.e.\ swapping the $i$- and $j$-loops.
        \textit{Loop Fusion} can be applied to increase data locality, i.e.\
        performing the computations of two subsequent for-loops that rely on
        the same data in a single for-loop instead to reduce cache misses.
        \item Every grid should be allocated in its entirety before moving
        on to the next one as this prevents cache thrashing.
        This is an example where \textit{Loop Fusion} should not be
        applied as the different for-loops operate on different data.
        \item As smoothing on the $0$-th grid takes the most time, the
        maximum possible speedup can be obtained by optimizing the
        \textbf{applyJacobi} method.
        Thus, to determine the effect of \textit{Loop Blocking}, the
        for-loops in \textbf{applyJacobi} can be replaced with the blocked
        for-loops:
\begin{lstlisting}[language=c++, basicstyle=\tiny]
for (int ii = 1; ii < g[l].N - 1; ii += B) {
  for (int jj = 1; jj < g[l].N - 1; jj += B) {
    for (int i = ii; i < std::min(ii + B, (int)g[l].N - 1); ++i) {
      for (int j = jj; j < std::min(jj + B, (int)g[l].N - 1); ++j) {
        g[l].U[i][j] =
            (g[l].Un[i - 1][j] + g[l].Un[i + 1][j] + g[l].Un[i][j - 1] +
             g[l].Un[i][j + 1] + g[l].f[i][j] * g_l_h_squared) *
            0.25;
      }
    }
  }
}
\end{lstlisting}
        However, even when trying different block sizes
        $B \in \left\{ 8, 16, 32, 64, 128 \right\}$ (larger block
        sizes are not applicable due to the size of the largest grid), the
        running time does not decrease.
        In fact, since the grid sizes are so small that enough data can fit
        into the cache, the overhead induced by the two additional for-loops
        causes the program to run slower.
        For this reason, \textit{Loop Blocking} was not used for the final
        implementation.
    \end{enumerate}

    \subsection*{Task 6}
    \label{subsec:Task6}
    \begin{enumerate}[a)]
        \item \begin{itemize}
                  \item Division operator replaced with multiplication of
                  inverse.
                  \item Multiplication by $1$ removed.
                  \item $pow\left(x, 2\right)$ replaced with $x \cdot x$.
                  \item Constants pre-computed and moved out of loops.
                  \item Expressions simplified with associativity.
                  \item Copy of array avoided by pointer swap.
        \end{itemize}
    \end{enumerate}

    \subsection*{Task 7}
    \label{subsec:Task7}
    To generate the optimization reports, the following line in the Makefile
    has to be changed to:
    \begin{lstlisting}[language=bash, basicstyle=\tiny]
CFLAGS = -O3 -D NOALIAS -D NOFUNCCALL -qopt-report=3 -qopt-report-phase=vec
    \end{lstlisting}

    \begin{enumerate}[a)]
        \item Most loops cannot be vectorized due to vector dependence.
        This means that the compiler cannot know if the data accessed in one
        iteration are also changed in another iteration (which would case a
        race condition if vectorized).
        \item This problem can be fixed by adding
        \lstinline[language=c++, basicstyle=\footnotesize]{#pragma ivdep}
        in front of the loop if the vectors are in fact independent (which
        they are as the results are computed from and stored into different
        vectors).
        \item SIMD operations start working at the beginning of a block of
        memory and thus if the arrays are not aligned with the block starts
        the operations need to gather data from different blocks which is
        very inefficient.
        During the exercise class the following grids allocation was
        recommended:
\begin{lstlisting}[language=c++, basicstyle=\scriptsize]
g[i].U = (double**) _mm_malloc (sizeof(double*) * g[i].N, 16);
for (int j = 0; j < g[i].N; j++) {
    g[i].U[j] = (double*) _mm_malloc (sizeof(double) * g[i].N, 16);
}
\end{lstlisting}
        However, even in combination with
        \lstinline[language=c++, basicstyle=\footnotesize]{#pragma vector aligned}
        this allocation slowed down the implementation (probably because it
        only aligns pointers and not the underlying data).
        Hence, the original allocation was used for the final implementation.
    \end{enumerate}

    \begin{table}[H]
        \caption{Iterations and running times for the three problems after
        all (beneficial) optimizations have been performed.}
        \begin{center}
            \begin{tabular}{c|ccc}
\toprule
 Problem & Iterations & Running Time & Verification \\
\midrule
       1 &       7844 &       1.422s &       Passed \\
       2 &      13467 &       8.238s &       Passed \\
       3 &      12865 &       7.824s &       Passed \\
\bottomrule
\end{tabular}

        \end{center}
        \label{table:FinalTimes}
    \end{table}

\end{document}
