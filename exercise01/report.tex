\documentclass[11pt]{article}


\usepackage{amsmath}
\usepackage{booktabs}
\usepackage{cleveref}
\usepackage{enumerate}
\usepackage{float}
\usepackage{graphicx}
\usepackage{listings}

\setlength\parindent{0pt}

\begin{document}

    \title{HPCSE II - Exercise 1}
    \author{Anian Ruoss}
    \maketitle

    All experiments were run on a node obtained by running the following
    command:
    \begin{lstlisting}[language=bash, basicstyle=\footnotesize]
bsub -n 24 -R fullnode -R "select[model=XeonE5_2680v3]" -Is bash
    \end{lstlisting}

    \subsection*{Task 4}
    \label{subsec:Task4}
    \begin{enumerate}[a)]
        \item See the first row in Table~\ref{table:OptGridCount}.
        \item The optimal number of grids is $6$ as illustrated
        in Table~\ref{table:OptGridCount}.
        \begin{table}[H]
            \caption{Iterations and running times for different numbers of
            grids for DownRelaxations $=1$ and UpRelaxations $=1$.}
            \begin{center}
                \begin{tabular}{c|ccc}
\toprule
 GridCount & Iterations & Running Time & Verification \\
\midrule
         1 &     199204 &     150.165s &       Passed \\
         2 &      43522 &      57.425s &       Passed \\
         3 &      32457 &      46.268s &       Passed \\
         4 &      32102 &      46.524s &       Passed \\
         5 &      30625 &      44.926s &       Passed \\
         6 &      22973 &      33.810s &       Passed \\
         7 &      33171 &      48.768s &       Passed \\
         8 &      51762 &      75.945s &       Passed \\
\bottomrule
\end{tabular}

            \end{center}
            \label{table:OptGridCount}
        \end{table}
        \item The optimal combination of up and down relaxations is
        determined in a greedy way.
        The optimal number of down relaxations in terms of running time is $3$
        as illustrated in Table~\ref{table:OptDownRelax}.
        Although, values of $5$ and $7$ produce less iterations, the time per
        iteration is significantly higher.
        \begin{table}[H]
            \caption{Iterations and running times for different numbers of
            down relaxations given GridCount $= 6$ and UpRelaxations $= 1$.}
            \begin{center}
                \begin{tabular}{c|ccc}
\toprule
 DownRelaxations & Iterations & Running Time & Verification \\
\midrule
               1 &      22973 &      33.770s &       Passed \\
               2 &      37152 &      72.471s &       Passed \\
               3 &      13467 &      33.071s &       Passed \\
               4 &      20885 &      61.631s &       Passed \\
               5 &      12068 &      41.526s &       Passed \\
               6 &      14811 &      58.096s &       Passed \\
               7 &      10130 &      44.835s &       Passed \\
\bottomrule
\end{tabular}

            \end{center}
            \label{table:OptDownRelax}
        \end{table}
        The optimal number of up relaxations is $1$ as illustrated in
        Table~\ref{table:OptUpRelax}.
        \begin{table}[H]
            \caption{Iterations and running times for different numbers of up
            relaxations with GridCount $= 6$ and DownRelaxations $= 3$.}
            \begin{center}
                \begin{tabular}{c|ccc}
\toprule
 UpRelaxations & Iterations & Running Time & Verification \\
\midrule
             1 &      13467 &      33.055s &       Passed \\
             2 &      13467 &      34.066s &       Passed \\
             3 &      13467 &      35.200s &       Passed \\
             4 &      13467 &      36.909s &       Passed \\
             5 &      13467 &      37.349s &       Passed \\
             6 &      13467 &      38.467s &       Passed \\
             7 &      13467 &      39.613s &       Passed \\
\bottomrule
\end{tabular}

            \end{center}
            \label{table:OptUpRelax}
        \end{table}
        \item The optimal configuration is given by GridCount $= 6$,
        \mbox{DownRelaxations $= 3$}, and UpRelaxations $= 1$.
        \item The solution reaches its equilibrium state faster on coarser
        grids as every cell covers more area.
        Fine grids ensure that the discretization error remains small.
        \item TODO
    \end{enumerate}

    \subsection*{Task 5}
    \label{subsec:Task5}

    \begin{enumerate}[a)]
        \item \textit{Loop Interchange} can be applied to a prevent bad
        memory access patterns, i.e.\ swapping the $i$- and $j$-loops.
        \textit{Loop Fusion} can be applied to prevent the overhead of a
        for-loop, i.e.\ when two independent computations with the same
        iteration space are merged into a single for-loop.
        \item Every grid should be allocated in its entirety before moving
        on to the next one as this prevents cache thrashing.
        \item Omitted.
    \end{enumerate}

    \subsection*{Task 6}
    \label{subsec:Task6}
    \begin{enumerate}[a)]
        \item \begin{itemize}
                  \item Division operator replaced with multiplication of
                  inverse.
                  \item Multiplication by $1$ removed.
                  \item $pow\left(x, 2\right)$ replaced with $x \cdot x$.
                  \item Constants pre-computed and moved out of loops.
                  \item Expressions simplified with associativity.
                  \item Copy of array avoided by pointer swap.
        \end{itemize}
    \end{enumerate}

    \subsection*{Task 7}
    \label{subsec:Task7}
    \begin{enumerate}[a)]
        \item Most loops cannot be vectorized due to vector dependence.
        \item This can be fixed by adding
        \lstinline[language=c++, basicstyle=\footnotesize]{#pragma ivdep}
        before the loop if the vectors are in fact independent.
        \item Unaligned SIMD operations need to gather from different
        locations, whereas aligned operations rely on the fact that the
        relevant data is stored contiguously in memory.
    \end{enumerate}

    \begin{table}[H]
        \caption{Iterations and running times for the three problems after
        all optimizations have been performed.}
        \begin{center}
            \begin{tabular}{c|ccc}
\toprule
 Problem & Iterations & Running Time & Verification \\
\midrule
       1 &       7844 &       1.422s &       Passed \\
       2 &      13467 &       8.238s &       Passed \\
       3 &      12865 &       7.824s &       Passed \\
\bottomrule
\end{tabular}

        \end{center}
        \label{table:FinalTimes}
    \end{table}

\end{document}