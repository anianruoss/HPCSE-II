\documentclass{article}

\usepackage{booktabs}
\usepackage{graphicx}
\usepackage{multirow}
\usepackage{listings}
\usepackage[capitalise]{cleveref}

\setlength\parindent{0pt}

\begin{document}
    \title{HPCSE II - Exercise 6}
    \author{Anian Ruoss}
    \maketitle

    \section*{Task 1}
    \label{sec:Task1}

    We parallelize the heat2D model for the single-grid case.
    For ease of communication with MPI we change the \lstinline{gridLevel}
    struct to store the grids with \lstinline{double *} instead of
    \lstinline{double **}.
    First, we establish the problem geometry with
    \lstinline{MPI_Dims_create()}, \lstinline{MPI_Cart_create()}, and
    \lstinline{MPI_Cart_shift()}.
    Since the grid size is not a multiple of two we have to pay attention to
    the boundary cases.
    Every rank allocates its own subgrids for \lstinline{U}, \lstinline{Un},
    \lstinline{Res}, and \lstinline{f}.
    The root rank then initializes the problem on the entire grid and
    distributes the initial values to each rank.
    Since this step is only performed once we do not employ custom data
    types even though this incurs some memory overhead due to data duplication.

    \subsection*{a)}
    \label{subsec:Task1a}

    After every Jacobi iteration we need to exchange the boundaries between
    neighboring ranks.
    Since we cannot pass the grid communicator to \lstinline{applyJacobi()}
    (as this method is declared in \lstinline{heat2d_mpi.hpp} where MPI is
    not initialized yet), we move the loop over the
    \lstinline{downRelaxations} out of \lstinline{applyJacobi()}.
    Next, we define data types for the contiguous and non-contiguous
    boundaries.
    Finally, we only need to adapt the loop boundaries to the subgrid sizes.

    \subsection*{b)}
    \label{subsec:Task1b}

    The \lstinline{calculateResidual()} method is trivially parallelizable
    by adapting the loop boundaries to the subgrid sizes.
    For \lstinline{calculateL2Norm()} we need to be careful not to loop over
    the ghost cells (we make use of the fact that the Dirichlet boundary
    condition is zero).
    After computing the sum of squared residuals for every subgrid we reduce
    the sum with MPI before computing the $L_2$-norm.

    \subsection*{Evaluation}
    \label{subsec:Task1Eval}

    We run the baseline sequential model and compare it with the execution
    times from running the parallelized model on one, two, and four full nodes
    on Euler.
    We display the running times
    in~\cref{lst:Baseline,lst:OneNode,lst:TwoNodes,lst:FourNodes}.

    \begin{lstlisting}[basicstyle=\scriptsize, caption={Output from running
    the baseline sequential model on one full node.}, label={lst:Baseline}]
        Time (s)     Grid0 Total
        -------------|---------|---------
        Smoothing | 570.226 | 570.226
        Residual | 204.614 | 204.614
        Restriction | 0.000 | 0.000
        Prolongation | 0.000 | 0.000
        L2Norm | 219.811 | 219.811
        -------------|---------|---------
        Total | 994.652 | 994.652
        -------------|---------|---------

        Running Time : 994.954s
        L2Norm: 70812.9164
    \end{lstlisting}
    \begin{lstlisting}[basicstyle=\scriptsize, caption={Output from running
    the parallelized model on one full node.}, label={lst:OneNode}]
        Time (s)    Grid0 Total
        -------------|---------|---------
        Smoothing | 18.228 | 18.228
        Residual | 6.370 | 6.370
        Restriction | 0.000 | 0.000
        Prolongation | 0.000 | 0.000
        L2Norm | 9.528 | 9.528
        -------------|---------|---------
        Total | 34.126 | 34.126
        -------------|---------|---------

        Running Time : 35.273s
        L2Norm: 70812.9164
    \end{lstlisting}
    \begin{lstlisting}[basicstyle=\scriptsize, caption={Output from running
    the parallelized model on two full nodes.}, label={lst:TwoNodes}]
        Time (s)    Grid0 Total
        -------------|---------|---------
        Smoothing | 8.896 | 8.896
        Residual | 3.271 | 3.271
        Restriction | 0.000 | 0.000
        Prolongation | 0.000 | 0.000
        L2Norm | 7.684 | 7.684
        -------------|---------|---------
        Total | 19.851 | 19.851
        -------------|---------|---------

        Running Time : 19.926s
        L2Norm: 70812.9164
    \end{lstlisting}
    \begin{lstlisting}[basicstyle=\scriptsize, caption={Output from running
    the parallelized model on two full nodes.}, label={lst:FourNodes}]
        Time (s)    Grid0 Total
        -------------|---------|---------
        Smoothing | 3.993 | 3.993
        Residual | 1.598 | 1.598
        Restriction | 0.000 | 0.000
        Prolongation | 0.000 | 0.000
        L2Norm | 6.009 | 6.009
        -------------|---------|---------
        Total | 11.600 | 11.600
        -------------|---------|---------

        Running Time : 13.509s
        L2Norm: 70812.9164
    \end{lstlisting}

    \newpage

    It can be observed that all runs compute the same final $L_2$-norm value.
    We display the strong scaling speed-ups and efficiencies
    in~\cref{tab:Runtimes}.
    Both \lstinline{applyJacobi()} and \lstinline{calculateResidual()}
    achieve speedups that exceed $100$\%.
    This is due to the fact that we do not include the communication time in
    the measured kernel time for simplicity of the implementation.
    Hence, it is more meaningful to analyze the speed-up and efficiency for
    the total running time.
    We observe that for one and two nodes we achieve perfect
    efficiency\footnote{We believe that the efficiency is slightly above
    $100$\% as we avoid the loop overhead by commenting out the prolongation-
    and restriction-loops since we only consider the single-grid case.},
    whereas we are probably limited by an increased communication overhead
    when running on four full nodes.
    Lastly, we note that the sharp efficiency decline for
    \lstinline{calculateL2Norm()} is due to the expensive collective
    \lstinline{MPI_Allreduce()} operation.

    \begin{table}
        \caption{Strong scaling analysis for the parallelized heat2D model.}
        \label{tab:Runtimes}
        \begin{center}
            \resizebox{.5\textwidth}{!}{ \begin{tabular}{cccc}
    \toprule
                                & Nodes & Speed-up & Efficiency (\%) \\
    \midrule
    \multirow{3}{*}{Smoothing}  & 1     & 31.3     & 130             \\
                                & 2     & 64.1     & 134             \\
                                & 4     & 142.8    & 149             \\
    \midrule
    \multirow{3}{*}{Residual}   & 1     & 32.1     & 134             \\
                                & 2     & 62.6     & 130             \\
                                & 4     & 128.0    & 133             \\
    \midrule
    \multirow{3}{*}{$L_2$-norm} & 1     & 23.1     & 96              \\
                                & 2     & 28.6     & 60              \\
                                & 4     & 36.6     & 38              \\
    \midrule
    \multirow{3}{*}{Total}      & 1     & 28.2     & 118             \\
                                & 2     & 49.9     & 104             \\
                                & 4     & 73.7     & 77              \\
    \bottomrule
\end{tabular}
 }
        \end{center}
    \end{table}

    \begin{table}
        \caption{}
        \label{tab:StrongScaling}
        \resizebox{\textwidth}{!}{ \begin{tabular}{ccccc}
    \toprule
    \textbf{Nodes} & \textbf{MPI Processes per Node} & \textbf{OpenMP Threads per MPI Process} & \textbf{MPI\_Waitall} & \textbf{Total Time} \\
    \midrule
    1 & 24 & 1 & 2.90 & 6.96 \\
    1 & 4 & 6 & 4.46 & 14.43 \\
    2 & 2 & 6 & 13.39 & 23.00 \\
    3 & 1 & 6 & 67.85 & 77.21 \\
    \bottomrule
\end{tabular}
 }
    \end{table}
    \begin{table}
        \caption{}
        \label{tab:WeakScaling}
        \resizebox{\textwidth}{!}{ \begin{tabular}{ccccc}
    \toprule
    \textbf{Nodes} & \textbf{MPI Processes per Node} & \textbf{OpenMP Threads per MPI Process} & \textbf{MPI\_Waitall} & \textbf{Total Time} \\
    \midrule
    1 & 24 & 1 & 2.90 & 6.96 \\
    1 & 2 & 12 & 7.89 & 18.35 \\
    2 & 2 & 12 & 20.79 & 26.64 \\
    4 & 2 & 12 & 11.96 & 15.20 \\
    \bottomrule
\end{tabular}
 }
    \end{table}

\end{document}
