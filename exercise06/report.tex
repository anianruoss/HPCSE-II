\documentclass[11pt]{article}

\usepackage{booktabs}
\usepackage{float}
\usepackage{graphicx}
\usepackage{listings}
\usepackage{multirow}
\usepackage[parfill]{parskip}
\usepackage[capitalise]{cleveref}

\begin{document}
    \title{HPCSE II - Exercise 6}
    \author{Anian Ruoss}
    \maketitle

    \section*{Task 1}
    \label{sec:Task1}

    We parallelize the heat2D model for the single-grid case.
    For ease of communication with MPI we change the \lstinline{gridLevel}
    struct to store the grids with \lstinline{double *} instead of
    \lstinline{double **}.
    First, we establish the problem geometry with
    \lstinline{MPI_Dims_create()}, \lstinline{MPI_Cart_create()}, and
    \lstinline{MPI_Cart_shift()}.
    Since the grid size is not a multiple of two we have to pay attention to
    the boundary cases.
    Every rank allocates its own subgrids for \lstinline{U}, \lstinline{Un},
    \lstinline{Res}, and \lstinline{f}.
    The root rank then initializes the problem on the entire grid and
    distributes the initial values to each rank.
    Since this step is only performed once we do not employ custom data
    types even though this incurs some memory overhead due to data duplication.

    \subsection*{a)}
    \label{subsec:Task1a}

    After every Jacobi iteration we need to exchange the boundaries between
    neighboring ranks.
    Since we cannot pass the grid communicator to \lstinline{applyJacobi()}
    (as this method is declared in \lstinline{heat2d_mpi.hpp} where MPI is
    not initialized yet), we move the loop over the
    \lstinline{downRelaxations} out of \lstinline{applyJacobi()}.
    Next, we define data types for the contiguous and non-contiguous
    boundaries.
    Finally, we only need to adapt the loop boundaries to the subgrid sizes.

    \subsection*{b)}
    \label{subsec:Task1b}

    The \lstinline{calculateResidual()} method is trivially parallelizable
    by adapting the loop boundaries to the subgrid sizes.
    For \lstinline{calculateL2Norm()} we need to be careful not to loop over
    the ghost cells (we make use of the fact that the Dirichlet boundary
    condition is zero).
    After computing the sum of squared residuals for every subgrid we reduce
    the sum with MPI before computing the $L_2$-norm.

    \subsection*{Evaluation}
    \label{subsec:Task1Eval}

    We run the baseline sequential model and compare it with the execution
    times from running the parallelized model on one, two, and four full nodes
    on Euler.
    We display the running times
    in~\cref{lst:Baseline,lst:OneNode,lst:TwoNodes,lst:FourNodes}.

    \begin{lstlisting}[basicstyle=\scriptsize, caption={Output from running
    the baseline sequential model on one full node.}, float, floatplacement=h,
    frame=single, label={lst:Baseline}]
    Time (s)     Grid0 Total
    -------------|---------|---------
    Smoothing | 570.226 | 570.226
    Residual | 204.614 | 204.614
    Restriction | 0.000 | 0.000
    Prolongation | 0.000 | 0.000
    L2Norm | 219.811 | 219.811
    -------------|---------|---------
    Total | 994.652 | 994.652
    -------------|---------|---------

    Running Time : 994.954s
    L2Norm: 70812.9164
    \end{lstlisting}
    \begin{lstlisting}[basicstyle=\scriptsize, caption={Output from running
    the parallelized model on one full node.}, float, floatplacement=h,
    frame=single, label={lst:OneNode}]
    Time (s)    Grid0 Total
    -------------|---------|---------
    Smoothing | 18.228 | 18.228
    Residual | 6.370 | 6.370
    Restriction | 0.000 | 0.000
    Prolongation | 0.000 | 0.000
    L2Norm | 9.528 | 9.528
    -------------|---------|---------
    Total | 34.126 | 34.126
    -------------|---------|---------

    Running Time : 35.273s
    L2Norm: 70812.9164
    \end{lstlisting}
    \begin{lstlisting}[basicstyle=\scriptsize, caption={Output from running
    the parallelized model on two full nodes.}, float, floatplacement=h,
    frame=single, label={lst:TwoNodes}]
    Time (s)    Grid0 Total
    -------------|---------|---------
    Smoothing | 8.896 | 8.896
    Residual | 3.271 | 3.271
    Restriction | 0.000 | 0.000
    Prolongation | 0.000 | 0.000
    L2Norm | 7.684 | 7.684
    -------------|---------|---------
    Total | 19.851 | 19.851
    -------------|---------|---------

    Running Time : 19.926s
    L2Norm: 70812.9164
    \end{lstlisting}
    \begin{lstlisting}[basicstyle=\scriptsize, caption={Output from running
    the parallelized model on two full nodes.}, float, floatplacement=h,
    frame=single, label={lst:FourNodes}]
    Time (s)    Grid0 Total
    -------------|---------|---------
    Smoothing | 3.993 | 3.993
    Residual | 1.598 | 1.598
    Restriction | 0.000 | 0.000
    Prolongation | 0.000 | 0.000
    L2Norm | 6.009 | 6.009
    -------------|---------|---------
    Total | 11.600 | 11.600
    -------------|---------|---------

    Running Time : 13.509s
    L2Norm: 70812.9164
    \end{lstlisting}

    It can be observed that all runs compute the same final $L_2$-norm value.
    We display the strong scaling speed-ups and efficiencies
    in~\cref{tab:Runtimes}.
    Both \lstinline{applyJacobi()} and \lstinline{calculateResidual()}
    achieve speedups that exceed $100$\%.
    This is due to the fact that we do not include the communication time in
    the measured kernel time for simplicity of the implementation.
    Hence, it is more meaningful to analyze the speed-up and efficiency for
    the total running time.
    We observe that for one and two nodes we achieve perfect
    efficiency\footnote{We believe that the efficiency is slightly above
    $100$\% as we avoid the loop overhead by commenting out the prolongation-
    and restriction-loops since we only consider the single-grid case.},
    whereas we are probably limited by an increased communication overhead
    when running on four full nodes.
    Lastly, we note that the sharp efficiency decline for
    \lstinline{calculateL2Norm()} is due to the expensive collective
    \lstinline{MPI_Allreduce()} operation.

    \begin{table}
        \caption{Strong scaling analysis for the parallelized heat2D model.}
        \label{tab:Runtimes}
        \vspace{.25cm}
        \begin{center}
            \resizebox{.5\textwidth}{!}{ \begin{tabular}{cccc}
    \toprule
                                & Nodes & Speed-up & Efficiency (\%) \\
    \midrule
    \multirow{3}{*}{Smoothing}  & 1     & 31.3     & 130             \\
                                & 2     & 64.1     & 134             \\
                                & 4     & 142.8    & 149             \\
    \midrule
    \multirow{3}{*}{Residual}   & 1     & 32.1     & 134             \\
                                & 2     & 62.6     & 130             \\
                                & 4     & 128.0    & 133             \\
    \midrule
    \multirow{3}{*}{$L_2$-norm} & 1     & 23.1     & 96              \\
                                & 2     & 28.6     & 60              \\
                                & 4     & 36.6     & 38              \\
    \midrule
    \multirow{3}{*}{Total}      & 1     & 28.2     & 118             \\
                                & 2     & 49.9     & 104             \\
                                & 4     & 73.7     & 77              \\
    \bottomrule
\end{tabular}
 }
        \end{center}
    \end{table}

    \section*{Task 2}
    \label{sec:Task2}

    \subsection*{a)}
    \label{subsec:Task2a}

    We implement the first hybrid approach with MPI and OpenMP by adding
    \lstinline{#pragma omp parallel for collapse(3)} in front of the jacobi
    step loop.
    This does not induce a race condition since all iterations are
    independent from each other.

    \subsection*{Evaluation}
    \label{subsec:Task2Eval}

    To avoid inconsistent $L_2$-norm values when dividing the grid by $12$ or
    $24$ MPI ranks, we increase the grid size to $768$.
    We use the following command to run hybrid jobs on Euler with $N$ nodes,
    $M$ MPI ranks and $T$ OpenMP threads:

    \begin{lstlisting}[basicstyle=\tiny]
    bsub -R fullnode -n 24*N
    "unset LSB_AFFINITY_HOSTFILE;
    OMP_NUM_THREADS=T mpirun -n M --map-by node:PE=T --report-bindings ./hybrid"
    \end{lstlisting}

    We first try to gauge the performance loss incurred by network
    communication.
    To this end, we limit the total number of processes to $24$ and measure
    the running times when distributing these processes across MPI/OpenMP
    and multiple nodes.
    We display the running times in~\cref{tab:StrongScaling}.

    \begin{table}
        \caption{Comparison of intra-node and network communication effects.}
        \label{tab:StrongScaling}
        \vspace{.25cm}
        \begin{center}
            \resizebox{\textwidth}{!}{ \begin{tabular}{ccccc}
    \toprule
    \textbf{Nodes} & \textbf{MPI Processes per Node} & \textbf{OpenMP Threads per MPI Process} & \textbf{MPI\_Waitall} & \textbf{Total Time} \\
    \midrule
    1 & 24 & 1 & 2.90 & 6.96 \\
    1 & 4 & 6 & 4.46 & 14.43 \\
    2 & 2 & 6 & 13.39 & 23.00 \\
    3 & 1 & 6 & 67.85 & 77.21 \\
    \bottomrule
\end{tabular}
 }
        \end{center}
    \end{table}

    The first row displays our baseline which uses $24$ MPI ranks distributed
    evenly among both sockets of one node.
    Hence, our baseline relies on intra-node communication only.
    The second row displays the output from running our hybrid code with $4$
    MPI ranks with $6$ OpenMP threads each, all running on the same node.
    The first two ranks are assigned to the first socket, the last two ranks
    to the second socket.
    We observe that --- although the number of computed entries increases
    by a factor of $6$ for every rank --- the total compute time increases
    only by a factor of $\approx 2.2$ meaning that we achieve an efficiency
    of $\approx 45$\% when considering only OpenMP.\@
    Compared to the baseline, this hybrid model uses $6$ times fewer ranks
    which means that the number of communications decreases from $92$ to $8$.
    However, every rank has to exchange $\approx 3.5$ times more boundary
    entries and we observe that this incurs an increase in the overall
    \lstinline{MPI_Waitall} time for the hybrid intra-node case.
    This effect becomes even stronger when we distribute the MPI ranks
    across multiple nodes.
    The third row displays the execution times for a run with $4$ MPI ranks
    with $6$ OpenMP threads each, of which $2$ ranks are on a separate node.
    The last row displays the execution times for with $4$ MPI ranks
    with $6$ OpenMP threads each, where every rank is on a node of its own.
    For this case, where all communication is performed over the network, we
    observe a substantial increase in \lstinline{MPI_Waitall} time compared to
    the second row which relies on intra-node communication only.

    Next, we try to measure whether using a hybrid model has a positive effect
    on the total running time.
    We run the hybrid model with $2$ MPI ranks per node and $12$ OpenMP
    threads per rank, making sure that each rank is assigned to its own socket.
    As a consequence we reduce the amount of network communication between MPI
    ranks.
    We compare our hybrid model against a pure MPI model on one, two, and
    four nodes and display the running times in~\cref{tab:WeakScaling}.

    \begin{table}
        \caption{Runtime comparison of MPI and hybrid MPI/OpenMP models.}
        \label{tab:WeakScaling}
        \vspace{.25cm}
        \begin{center}
            \resizebox{\textwidth}{!}{ \begin{tabular}{ccccc}
    \toprule
    \textbf{Nodes} & \textbf{MPI Processes per Node} & \textbf{OpenMP Threads per MPI Process} & \textbf{MPI\_Waitall} & \textbf{Total Time} \\
    \midrule
    1 & 24 & 1 & 2.90 & 6.96 \\
    1 & 2 & 12 & 7.89 & 18.35 \\
    2 & 2 & 12 & 20.79 & 26.64 \\
    4 & 2 & 12 & 11.96 & 15.20 \\
    \bottomrule
\end{tabular}
 }
        \end{center}
    \end{table}

    It can be observed that for one and two nodes the increased message
    sizes --- due to the smaller number of MPI ranks --- supersedes the
    reduced cost of less communication over the network.
    However, for four nodes the \lstinline{MPI_Waitall} time is already
    smaller for the hybrid case and we would except the hybrid model to
    significantly outperform the MPI model on more nodes.

\end{document}
