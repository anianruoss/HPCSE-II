\documentclass[11pt]{article}

\usepackage{amsmath}
\usepackage{listings}

\setlength\parindent{0pt}

\begin{document}
    \title{HPCSE II - Exercise 4}
    \author{Anian Ruoss}
    \maketitle

    \section*{Task 1}
    \label{sec:Task1}

    As in Task 2 of Part 2 of Homework 3, we use the head2DSolver to model
    the temperature distribution on the steel sheets.
    For every torch we want to determine the optimal beam width, beam
    intensity and x- and y-coordinates meaning that we have 16 parameters in
    total for the 4 robotic torches.
    For all parameters we know the upper and the lower bounds given by:
    \begin{itemize}
        \item $x \in [0.0, 0.5]$ for torches 1 and 2
        \item $x \in [0.5, 1.0]$ for torches 3 and 4
        \item $y \in [0.0, 1.0]$ for all torches
        \item $\text{beam intensity} \in [0.4, 0.6]$ for all torches
        \item $\text{beam width} \in [0.04, 0.06]$ for all torches
    \end{itemize}
    Since we do not have any additional information about the parameter
    distributions, we model all parameters as uniformly distributed
    between within their respective bounds.
    We employ Korali's CMA-ES solver to maximize the posterior
    distribution of the parameters (even though we don't really have prior
    distributions, but the posterior was given in the template) to find the
    optimal parameter values and we display the results in
    Listing~\ref{lst:Task1}.

    \begin{lstlisting}[basicstyle=\tiny, frame=single, caption={Korali
    output when maximizing the likelihood of the heat distribution given the
    model with four candles.}, label={lst:Task1}]
    [Korali] Starting CMAES. Parameters: 17, Seed: 0xFFFFFFFFFFFFFFFF
    ...
    [Korali] Finished - Reason: Object variable changes < 1.00e-06
    [Korali] Parameter 'Sigma' Value: 0.937023
    [Korali] Parameter 'torch_1_x' Value: 0.242895
    [Korali] Parameter 'torch_1_y' Value: 0.240862
    [Korali] Parameter 'torch_1_intensity' Value: 0.505957
    [Korali] Parameter 'torch_1_width' Value: 0.046153
    [Korali] Parameter 'torch_2_x' Value: 0.251464
    [Korali] Parameter 'torch_2_y' Value: 0.741347
    [Korali] Parameter 'torch_2_intensity' Value: 0.479662
    [Korali] Parameter 'torch_2_width' Value: 0.051501
    [Korali] Parameter 'torch_3_x' Value: 0.757994
    [Korali] Parameter 'torch_3_y' Value: 0.254470
    [Korali] Parameter 'torch_3_intensity' Value: 0.474448
    [Korali] Parameter 'torch_3_width' Value: 0.054619
    [Korali] Parameter 'torch_4_x' Value: 0.760434
    [Korali] Parameter 'torch_4_y' Value: 0.770577
    [Korali] Parameter 'torch_4_intensity' Value: 0.475872
    [Korali] Parameter 'torch_4_width' Value: 0.059926
    [Korali] Total Elapsed Time: 234.156253s
    \end{lstlisting}

    Korali performs the optimization in roughly 4 minutes.
    We know from the lecture that CMA-ES is embarrassingly parallel since we
    can compute and evaluate every sample independently, which can be
    achieved by parallelizing the two for-loops at lines 11 and 14.
    Typically the evaluation is a lot more costly than random
    number generation, which is why we should already observe a considerable
    speedup when only parallelizing the loop at line 14.

    \section*{Task 2}
    \label{sec:Task2}

    We run the single tasking engine and display its output in
    Listing~\ref{lst:Task2Single}.

    \begin{lstlisting}[basicstyle=\tiny, frame=single, caption={Output from
    executing the single tasking engine.}, label={lst:Task2Single}]
    Processing 240 Samples each with 2 Parameter(s)...
    Verification Passed
    Total Running Time: 29.717s
    \end{lstlisting}

    \subsection*{a)}
    \label{subsec:Task2a}

    Since all samples are well-known at the beginning, they can be distributed
    evenly among all ranks and gathered back to one rank once the evaluations
    are completed.
    We implement this divide-and-conquer strategy with UPC++ and MPI and
    display the results obtained from running the implementations with 24
    ranks on an Euler compute node in listings~\ref{lst:Task2aUPCXX}
    and~\ref{lst:Task2aMPI} respectively.

    \begin{lstlisting}[basicstyle=\tiny, frame=single, caption={Output from
    executing the UPC++ tasking engine with the divide-and-conquer strategy.},
    label={lst:Task2aUPCXX}]
    Verification Passed
    Total time:           1.37665
    Average time:         1.17792
    Load imbalance ratio: 0.144362
    \end{lstlisting}

    \begin{lstlisting}[basicstyle=\tiny, frame=single, caption={Output from
    executing the MPI tasking engine with the divide-and-conquer strategy.},
    label={lst:Task2aMPI}]
    Verification Passed
    Total time:           1.31242
    Average time:         1.17451
    Load imbalance ratio: 0.10508
    \end{lstlisting}

    We observe a speedup of $\approx 21.5$ for UPC++ and $\approx 22.5$ for
    MPI and thus we report efficiencies of $\approx 90\%$ for UPC++ and
    $\approx 94.5\%$ for MPI.\@
    Both implementations suffer from a relatively high load imbalance ratio
    ($\approx 0.145$ for UPC++ and $\approx 0.105$ for MPI) which results
    from the fluctuation in evaluation times.
    The MPI approach is practically identical to the UPC++ code but the
    collective operations in MPI allow for a much cleaner implementation.
    In general, MPI feels more natural since it requires very explicit
    communication.

    \subsection*{b)}
    \label{subsec:Task2b}

    To solve the load imbalance problem observed in Task 2a) we implement
    the producer-consumer strategy which takes advantage of the fact that the
    evaluation times differ and distributes workloads
    according to rank availability and not according to a fixed scheme.
    We display the results obtained from running the UPC++ and MPI
    implementations with 24 ranks on an Euler compute node in
    listings~\ref{lst:Task2bUPCXX} and~\ref{lst:Task2bMPI} respectively.

    \begin{lstlisting}[basicstyle=\tiny, frame=single, caption={Output from
    executing the UPC++ tasking engine with the producer-consumer strategy.},
    label={lst:Task2bUPCXX}]
    Processing 240 Samples each with 2 Parameter(s)...
    Verification Passed
    Total time:           1.32383
    Average time:         1.25486
    Load imbalance ratio: 0.0520984
    \end{lstlisting}

    \begin{lstlisting}[basicstyle=\tiny, frame=single, caption={Output from
    executing the MPI tasking engine with the producer-consumer strategy.},
    label={lst:Task2bMPI}]
    Verification Passed
    Total time:           1.32354
    Average time:         1.24246
    Load imbalance ratio: 0.0612577
    \end{lstlisting}

    We observe for both UPC++ and MPI that the load imbalance ratio drops
    significantly compared to the divide-and-conquer strategy, although more
    drastically for UPC++.
    For UPC++ we observe that the total time decreases slightly compared to
    the divide-and-conquer strategy (speedup: $\approx 22.5$, efficiency:
    $\approx 93.5\%$), whereas for the MPI implementation the total running
    time increases marginally (speedup: $\approx 22.5$, efficiency:
    $\approx 93.5\%$).
    Whereas the MPI implementation outperformed UPC++ for the
    divide-and-conquer strategy, they are now practically identical.
    The MPI approach differs slightly from the UPC++ approach:
    \begin{itemize}
        \item UPC++ employs a queue of consumers which contain a future among
        other data.
        The producer iterates over the queue and checks whether a RPC has
        completed before distributing another sample to the idling rank.
        \item The MPI implementation does not require a queue as the
        producer just sends samples and listens for results until all
        samples have been evaluated.
        Unlike the UPC++ implementation, we need to explicitly tell every
        rank that the evaluation has completed once all samples have been
        processed.
    \end{itemize}
    Even though we were told that the producer-consumer problem would be
    easier to implement in UPC++ than in MPI, the MPI approach feels cleaner
    as it is more explicit\footnote{From \emph{The Zen of Python, by Tim
    Peters}: ``Explicit is better than implicit.``.}.

    \section*{Task 3}
    \label{sec:Task3}

    We run the single tasking engine and display its output in
    Listing~\ref{lst:Task3Single}.

    \begin{lstlisting}[basicstyle=\tiny, frame=single, caption={Output from
    executing the single tasking engine.}, label={lst:Task3Single}]
    Processing 240 Samples (24 initially available), each with 2 Parameter(s)...
    Verification Passed
    Total Running Time: 29.458s
    \end{lstlisting}

    Since not all samples are available at the beginning of the generation
    it does not make sense to have more ranks than initially available
    samples and we enforce this constraint with an assert.
    Apart from that our approaches are similar to those from Task2b) and we
    display the results obtained from running the UPC++ and MPI
    implementations with 24 ranks on an Euler compute node in
    listings~\ref{lst:Task3UPCXX} and~\ref{lst:Task3MPI} respectively.

    \begin{lstlisting}[basicstyle=\tiny, frame=single, caption={Output from
    executing the UPC++ tasking engine .},
    label={lst:Task3UPCXX}]
    Processing 240 Samples (24 initially available), each with 2 Parameter(s)...
    Verification Passed
    Total Running Time: 1.355s
    \end{lstlisting}

    \begin{lstlisting}[basicstyle=\tiny, frame=single, caption={Output from
    executing the MPI tasking engine .},
    label={lst:Task3MPI}]
    Verification Passed
    Total Running Time: 1.311s
    \end{lstlisting}

    We observe speedups of $\approx 21.5$ for UPC++ and $\approx 22.5$ for
    MPI and correspondingly efficiencies of $\approx 90.5\%$ for UPC++ and
    $\approx 93.5\%$ for MPI.\@
    One challenge we faced during the implementation of this exercise is
    that \mbox{\textit{getSample()}} and \mbox{\textit{updateEvaluation()}}
    have to be called from the root rank and that updateEvaluation has to be
    called after the evaluation has completed on a consumer rank.
    This problem can be elegantly solved with the \textit{then()} method
    from UPC++ which executes a function on root after the RPC has completed.
    Our MPI implementation is basically equivalent to that of Task2b) and
    thus we refer to Task2b) for the discussion of the difference of the two
    approaches.
    Using the $then()$ function is definitely more elegant than sending data
    back and forth as is required for MPI.\@


\end{document}